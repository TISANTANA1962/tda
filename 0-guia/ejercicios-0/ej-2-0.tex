\begin{enunciado}{\ejercicio}
  Encontrar una fórmula para la siguiente suma y
  demostrarla por inducción:
  $$
    1 + 2 + 2^2 + 2^3 + \ldots + 2^n.
  $$
\end{enunciado}
Es la geométrica:
$$
  \sumatoria{i = 0}{n} 2^i = \frac{2^{i + 1} - 1}{2 - 1} = 2^{i + 1} - 1
$$

Por inducción quiero probar que la siguiente proposición sea verdadera:
$$
  p(n) :
  \sumatoria{i = 0}{n} 2^i =
  \frac{2^{i + 1} - 1}{2 - 1} \quad \paratodo n \en \naturales
$$

\textit{Caso base}
$$
  p(\blue{1}) :
  \sumatoria{i = 0}{\blue{1}} 2^i = 1 + 2 = 3 = 2^{\blue{1} + 1} - 1
$$
La proposición $p(\blue{1})$ resultó ser verdadera.

\textit{Paso inductivo}
Para algún valor $k \en \naturales$ asumo que la proposición:
$$
  p(\blue{k}) :
  \ub{
    \sumatoria{i = 0}{\blue{k}} 2^i = 2^{\blue{k} + 1} - 1
  }{
    \text{\purple{hipótesis inductiva}}
  }
$$
es verdadera. Quiero probar entonces que:
$$
  p(\blue{k+1}) :
  \sumatoria{i = 0}{\blue{k+1}} 2^i = 2^{\blue{k+1} + 1} - 1,
$$
también lo sea.

Usando el viejo, querido y confiable truquito:
$$
  \sumatoria{i = 0}{\blue{k+1}} 2^i =
  2^{k+1} + \sumatoria{i = 0}{\blue{k}} 2^i
  \igual{\purple{HI}}
  2^{k+1} + 2^{\blue{k} + 1} - 1 = 2^{k+2} - 1.
$$
Resultó que la proposición $p(k+1)$ también es verdadera.

Entonces como $p(1), p(k) \text{ y } p(k+1)$ son verdaderas por el principio de inducción también lo es $p(n) \paratodo n \en \naturales$.

\fin
