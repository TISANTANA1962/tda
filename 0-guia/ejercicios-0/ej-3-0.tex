\begin{enunciado}{\ejercicio}
  La población de una colonia de hormigas se duplica todos los años. Si se establece una colonia inicial de 10 hormigas,
  ¿Cuántas hormigas habrá después de $n$ años?
\end{enunciado}

La función que describe el crecimiento de la población:
$$
  H(t) = 10 \cdot 2^t
$$
Después de $t = n$ años habrá $10 \cdot 2^{n}$ hormigas. ¿Habría que probar esto por inducción o algo así? \textit{Pajilla}

\fin
