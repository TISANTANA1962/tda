\setcounter{ejercicio}{4}
\begin{enunciado}{\ejercicio}
  La población de gatos en un depósito tiene la propiedad de que el número de gatos en un año es
  igual a la suma del número de gatos de los dos años anteriores. Si en el primer año (empezando
  a contar desde 1) había un solo gato, y en el segundo dos (suponiendo ello posible!), probar que
  el número de gatos en el año $n$ es:
  $$
    \textstyle
    \sqrt{\frac{1}{5}} \times
    \parentesis{
      \parentesis{
        \frac{1 + \sqrt{5}}{2}
      }^{n+1} -
      \parentesis{
        \frac{1 - \sqrt{5}}{2}
      }^{n+1}
    }
  $$
\end{enunciado}

La cantidad de gatos del año $n$, $G_n$ es la suma de los gatos de los dos años anteriores:
$$
  \llave{l}{
    G_1 = 1\\
    G_2 = 2
  }
  \ytext
  G_n \igual{def} G_{n - 1} + G_{n - 2}
$$
Esto es fibonacci y esos números son el número de oro:
$$
  \llave{rcl}{
    \varphi & = & \frac{1 + \sqrt{5}}{2} \\
    -\frac{1}{\varphi} & = & \frac{1 - \sqrt{5}}{2}
  }
  \flecha{mágicas}[propiedades]
  \llamada1
  \llave{rcl}{
    \varphi + \frac{1}{\varphi} & = & \sqrt{5}\\
    \varphi - \frac{1}{\varphi} & = & 1\\
    \varphi^n & = & \varphi^{n-1} + \varphi^{n-2}\\
    (\frac{-1}{\varphi})^n & = & (\frac{-1}{\varphi})^{n-1} + (\frac{-1}{\varphi})^{n-2}
  }
$$

Quiero probar que la proposición es verdadera:
$$
  \textstyle
  p(n) : G_n =
  \sqrt{\frac{1}{5}} \cdot
  \parentesis{
    \varphi^{n+1} -
    \parentesis{
      -\frac{1}{\varphi}
    }^{n+1}
  }
$$
Esto se puede probar por inducción:

\textit{Caso base:}
$$
  \textstyle
  p(\blue{1}) :
  G_{\blue{1}} =
  \frac{1}{\sqrt{5}} \cdot
  \parentesis{
    \varphi^2 -
    \parentesis{
      -\frac{1}{\varphi}
    }^2
  } =
  \frac{1}{\sqrt{5}} \cdot
  (
  \varphi - \frac{1}{\varphi}
  )
  \cdot
  (
  \varphi + \frac{1}{\varphi}
  )
  \igual{$\llamada1$}
  1
$$
Por lo que la proposición $p(\blue{1})$ resultó verdadera. Ahora pruebo el segundo caso base:
$$
  \textstyle
  p(\blue{2}) :
  G_{\blue{2}} =
  \frac{1}{\sqrt{5}} \cdot
  (
  \varphi^3 - ( -\frac{1}{\varphi} )^3
  )
$$
Cuentas feas del \textit{número de oro} usando mucho la propiedad de recursividad de potencias de $\varphi$ y $-\frac{1}{\varphi}$:
$$
  \begin{array}{rcl}
    \frac{1}{\sqrt{5}} \cdot ( \varphi^3 - (\frac{-1}{\varphi})^3)
     & \igual{\red{!!}}[$\llamada1$]            &
    \frac{1}{\sqrt{5}} \cdot (2\varphi + 1 - (\frac{-2}{\varphi} + 1)) \\
     & =                                        &
    \frac{1}{\sqrt{5}} \cdot
    (\frac{2\varphi^2 + 2}{\varphi})                                   \\
     & \igual{$\llamada1$}                      &
    \frac{1}{\sqrt{5}} \cdot
    (\frac{2(\varphi + 1) + 2}{\varphi})                               \\
     & =                                        &
    \frac{1}{\sqrt{5}} \cdot
    (\frac{2\varphi + 4}{\varphi})                                     \\
     & \igual{\red{!}}                          &
    \frac{1}{\sqrt{5}} \cdot
    (\frac{5 + \sqrt{5}}{\varphi})                                     \\
     & =                                        &
    \frac{\sqrt{5} + 1}{\varphi}                                       \\
     & \igual{\href{\mindExplosion}{\surprise}} &
    \frac{\sqrt{5} + 1}{\frac{1 + \sqrt{5}}{2}} = 2                    \\
  \end{array}
$$
Por lo tanto la proposición $p(\blue{2})$ es verdadera también.

\medskip

\textit{Paso inductivo:}

Asumo que para algún $k \en \naturales$ las proposiciones:
$$
  \textstyle
  p(\blue{k}) :
  \ub{
    G_{\blue{k}} =
    \sqrt{\frac{1}{5}} \cdot
    (
    \varphi^{k+1} - (-\frac{1}{\varphi})^{k+1}
    )
  }{
    \text{\purple{hipótesis inductiva}}
  }
  \ytext
  p(\blue{k+1}) :
  \ub{
    G_{\blue{k+1}} =
    \sqrt{\frac{1}{5}} \cdot
    (
    \varphi^{k+2} - (-\frac{1}{\varphi})^{k+2}
    )
  }
  {
    \text{\purple{hipótesis inductiva}}
  }
$$
son verdaderas, por lo tanto quiero probar que la proposición:
$$
  \textstyle
  p(\blue{k+2}) : G_{\blue{k+2}} =
  \sqrt{\frac{1}{5}} \cdot
  (
  \varphi^{k+3} -
  (-\frac{1}{\varphi}
  )^{k+3}
  )
$$
también sea verdadera.

Usando la definción de $G_n$ y las hipótesis inductivas, sale enseguida:
$$
  \begin{array}{rcl}
    G_{k+2}
     & \igual{def}         &
    G_{\blue{k+1}} + G_{\blue{k}} \\
     & =                   &
    \sqrt{\frac{1}{5}} \cdot
    \parentesis{
      \varphi^{k+1} - (-\frac{1}{\varphi})^{k+1}
      +
      \varphi^{k+2} - (-\frac{1}{\varphi})^{k+2}
    }                             \\
     & =                   &
    \sqrt{\frac{1}{5}} \cdot
    \parentesis{
      \parentesis{
        \varphi^{k+2} + \varphi^{k+1}
      }
      -
      \parentesis{
        (-\frac{1}{\varphi})^{k+2} + (-\frac{1}{\varphi})^{k+1}
      }
    }                             \\
     & \igual{$\llamada1$} &
    \sqrt{\frac{1}{5}} \cdot
    \parentesis{
      \varphi^{k+3}  - (-\frac{1}{\varphi})^{k+3}
    }
  \end{array}
$$
De manera que la proposición $p(k+1)$ resultó también verdadera.

\bigskip

Dado que $p(1),\, p(2),\, p(k),\, p(k+1) \ytext p(k+2)$ son todas verdaderas, por principio de inducción,
$p(n)$ también los será para todo $n \en \naturales$.

\fin
