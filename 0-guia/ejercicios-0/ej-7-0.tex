\setcounter{ejercicio}{6}
\begin{enunciado}{\ejercicio}
  ¿Cuál es el error en la siguiente demostración?

  Se quiere probar que los elementos $x_1, x_2, x_3,\ldots, x_n$ de un conjunto son iguales entre sí.
  \begin{enumerate}[label=\alph*)]
    \item Paso inicial ($n=1$): El conjunto tiene un sólo elemento $x_1$ que es igual a sí mismo.

    \item Paso inductivo: Supongamos que $x_1 = x_2 = x_3 = \cdots = x_{n-1}$. Cómo también
          vale la hipótesis inductiva para un conjunto de dos elementos, tenemos que $x_{n-1} = x_n$ y
          por tanto resulta $x_1 = x_2 = x_3 = \cdots x_{n-1} = x_n$.
  \end{enumerate}
\end{enunciado}

¿La igualdad de los elementos es independiente del tamaño del conjunto?

\red{consultar}
