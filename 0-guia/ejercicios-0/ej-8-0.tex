\begin{enunciado}{\ejercicio}
  ¿Cuál es el error en la siguiente demostración?

  Se quiere probar que $\paratodo a \distinto 0$ vale que $a^n = 1$.
  \begin{enumerate}[label=\alph*)]
    \item Paso inicial ($n = 0$): $a^n = 1 \paratodo a^n = 1$.
    \item Paso inductivo: Supongamos que $a^{n - 1} = 1$. Entonces $a^n = (a^{n-1} \times a^{n - 1})/a^{n-2} = (1 \times 1)/1 = 1$
  \end{enumerate}
\end{enunciado}

Que no da ninguna cuenta?
Que al asumir que $a^{n - 1} = 1 \sii (n = 1 \lor a = 1)$

\red{consultar}

