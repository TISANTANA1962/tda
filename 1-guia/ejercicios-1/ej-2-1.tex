\begin{enunciado}{\ejercicio[BusquedaBinaria]}
  \textit{BusquedaBinaria}

  Dado el algoritmo de \textit{búsqueda binaria}, implementado en el siguiente código \python:
  \codigoPython{ej-2/codigo1-1.py}

  \begin{enumerate}[label=\arabic*)]
    \item Identificar qué líneas son el \textit{divide}, cuáles son el \textit{conquer} y cuáles son el \textit{combine}.

    \item ¿En cuántos subproblemas se divide?

    \item ¿De qué tamaño son estos subproblemas?

    \item ¿Cuál es el costo de combinar los resultados de los subproblemas?

    \item Escribir la función $T(n)$ de manera recursiva.

    \item Determinar la complejidad del algoritmo utilizando el Teorema Maestro.
  \end{enumerate}
\end{enunciado}

\begin{enumerate}[label=\arabic*)]
  \item
        \begin{itemize}
          \item \textit{Divide:} Cuando divido al \texttt{input} a la mitad en la \ul{línea 4} y también
                en la comparación de la línea 7, donde se decide por cual mitad seguir.

          \item \textit{Conquer:} Llamada de recursión en las \ul{líneas 8 y 10}.

          \item \textit{Combine:} En este algoritmo no hay \textit{combine}.
        \end{itemize}

  \item El árbol de recursión solo tiene una rama, porque hay solo una llamada recursiva en cada llamada.
        Cada nodo tendrá la mitad del tamaño del input anterior.

  \item El tamaño de cada subproblema será de $\frac{n}{\blue{2}} > n_0 = 1$.

  \item En este algoritmo no hay un \textit{combine} de unir los resultados como en el \texttt{merge sort}.
        La complejidad de devolver \texttt{true} o \texttt{false} será $O(1)$. Hay solo unas comparaciones y
        el cálculo de \texttt{medio}, todo eso es independiente de $n$, todo $O(1)$

  \item $$
          T(n) =
          \llave{rcl}{
            1 &\text{si}& n \leq 1 \\
            T(\frac{n}{2}) + 1 &\text{si}& n > 1
          }
        $$
        Esa función recursiva equivale a contar la cantidad de nodos que tendrá el árbol de recursiones: Cada nodo
        tiene un costo 1 (o O(1), eh lo meesmo.) Por lo tanto sumo todo:
        $$
          T(n) \en \Theta(\log_2(n))
        $$

  \item
        El \textit{Teorema Maestro} me estima la complejidad total de T(n) según el
        valor de $f(n)$.

        En este caso $f(n) \en O(1)$ y como $\magenta{a} = 1$ y $\blue{b} = 2$:
        $$
          f(n) \en O(n^{\log_{\blue{b}}(\magenta{a})} \cdot \log^k(n))
          \Entonces{$\magenta{a} = 1,\, \blue{b} = 1$}[$k = 0$]
          f(n) \en O(1)
        $$
        Estoy en el caso 2 del teorema:
        $$
          T(n) \en \Theta(n^{\log_{\blue{b}}(\magenta{a})} \cdot \log^{k+1}(n)
          \Entonces{$\magenta{a} = 1,\, \blue{b} = 1$}[$k = 0$]
          \cajaResultado{
            T(n) \en \Theta(\log(n))
          }
        $$
\end{enumerate}

\fin
