\def\tv{\green{\textit{travesíaVital}}}
\def\posicionValida{\green{\textit{posiciónVálida}}}
\def\vidaValida{\green{\textit{vidaVálida}}}
\begin{enunciado}{\ejercicio[Travesía Vital]}
  \textit{Travesía Vital}

  Hay un terreno, que podemos pensarlo como una grilla de $m$ filas y $n$ columnas, con trampas
  y pociones. Queremos llegar de la esquina superior izquierda hasta la inferior derecha, y desde
  cada casilla sólo podemos movernos a la casilla de la derecha o a la de abajo. Cada casilla $i,j$
  tiene un número entero $A_{i,j}$ que nos modificará el nivel de vida sumándonos el número $A_{i,j}$ (si
  es negativo, nos va a restar $|A_{i,j}|$ de vida).
  Queremos saber el mínimo nivel de vida con el que debemos comenzar tal que haya un camino posible de
  modo que en todo momento nuestro nivel de vida sea al menos 1. Por ejemplo, si tenemos la grilla
  $$
    A =
    \begin{bmatrix}
      -2 & -3  & 3  \\
      -5 & -10 & 1  \\
      10 & 30  & -5
    \end{bmatrix}
  $$
  el mínimo nivel de vida con el que podemos comenzar es 7 porque podemos realizar el camino
  que va todo a la derecha y todo abajo.
  \begin{enumerate}[label=$\alph*)$]
    \item Pensar la idea de un algoritmo de \textit{backtracking} (no hace falta escribirlo).

    \item\label{ej-16:item-b} Convencerse de que, excepto que estemos en los límites del terreno, la mínima vida necesaria
          al llegar a la posición $i, j$ es el resultado de restar al mínimo entre la mínima vida necesaria
          en $i+1,j$ y aquella en $i,j+1$, el valor $A_{i,j}$, salvo que eso fuera menor o igual que 0, en
          cuyo caso sería 1.

    \item Escribir una formulación recursiva basada en \ref{ej-16:item-b}. Explicar su semántica e indicar cuáles
          serían los parámetros para resolver el problema.

    \item Diseñar un algoritmo de PD y dar su complejidad temporal y espacial auxiliar. Comparar cómo resultaría un enforque \textit{top-down} con uno
          \textit{bottom-up}.

    \item
          Dar un algoritmo \textit{bottom-up} cuya complejidad temporal sea $O(m \cdot n)$ y la espacial auxiliar sea $O(\minimo(m,n))$.
  \end{enumerate}
\end{enunciado}

Buen enunciado, consignas guiadas haciendo foco en un procedimiento incremental. Contrasta fuertemente con esos ejercicios de verga
sacados de \textit{Codeforces} los cuales solo tienen sentido para practicar cuando uno ya "\textit{se siente cómodo}" con la técnica
\href{https://youtu.be/r1g5PoRSp1c?t=29}{...que difícil se me hace.}

\begin{enumerate}[label=$\alph*)$]
  \item ¿Pensar? Eso es la receta para entrar en una vorágine de caca recursiva. No pienso luego no existo, \textit{fuerza bruta}, como le gusta a tu hermana.

  \item Me costó implementar lo de "en cuyo caso sería 1", pero tiene sentido. Poner el 1 de prepo de está diciendo que la \textit{energía inicial} será
        lo que tenga que ser para llegar a esa celda y sobrevivir como un campeón.

  \item
        Ahí te va un primer intento fallido:
        Matriz $A \en \reales^{m \times n}$, global.
        $$
          \tv(i, j)
          =
          \llave{ccl}{
            1 & \text{si} & i > m \land  j > n \\
            - A_{i,j} + \tv(i+1, j) & \text{si} & i \leq m \land  j > n \\
            - A_{i,j} + \tv(i, j+1) & \text{si} & i > m \land  j \leq n \\
            - A_{i,j} +
            \minimo
            \matriz{c}{
              \tv(i+1, j),  \\
              \tv(i, j+1)
            }
            & \text{si} & i \leq m \land  j \leq n \\
          }
        $$
        Si bien está buena, está mal, ha! Esa función permite potencialmente que cualquier celda me dé una "\textit{energía acumulada}" menor a 1.
        Así que tengo que poner algo para que en ningún momento el camino me dé menor a 1:
        $$
          \tv(i, j)
          =
          \llave{ccl}{
            1 & \text{si} & i > m \land  j > n \\
            \maximo(1, - A_{i,j} + \tv(i+1, j)) & \text{si} & i \leq m \land  j > n \\
            \maximo(1, - A_{i,j} + \tv(i, j+1)) & \text{si} & i > m \land  j \leq n \\
            \maximo
            \matriz{c}{
              1,
              - A_{i,j} +
              \minimo
              \matriz{c}{
                \tv(i+1, j),  \\
                \tv(i, j+1)
              }
            }
            & \text{si} & i \leq m \land  j \leq n \\
          }
        $$

        La función devuelve el mínimo valor inicial de energía que hay que tener llegar a la posición $m,n$ sin tener nunca una suma acumulada menor a 1
        en todo el recorrido.
        Los parámetros $i,j$ representan las coordenadas en la matriz que se va recorriendo. Llamar a la función con $\tv(1,1)$ devuelve el valor buscado.

  \item \Hacer

  \item \Hacer
\end{enumerate}

