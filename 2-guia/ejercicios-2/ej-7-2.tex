\begin{enunciado}{\ejercicio[Dobra]}
  \textbf{\textit{Dobra}}

  Dobra se encuentra con muchas palabras en su vida, como es una persona particular la mayoría
  de estas no le gustan. Para compensar empezó a inventar palabras más agradables. Dobra crea
  palabras nuevas escribiendo una cadena de caracteres que considera buena, luego borra los caracteres que peor
  le caen y los reemplaza con \_.

  Luego para mejorar su vida intenta reemplazar estos guiones bajos con letras más aceptables
  intentando crear palabras más lindas. Dobra considera una palabra como buena si no contiene 3
  vocales consecutivas, 3 consonantes consecutivas y al menos contiene una E.

  \begin{enumerate}[label=$\alph*)$]
    \item Mostrar alguna solución candidata posible y alguna solución parcial.

    \item\label{ej7:item-b} Proponer una función recursiva y estimar su complejidad. Asumir que se tiene una función
          \green{\textit{verificar}} que toma una cadena y devuelve \true si es como Dobra quiere o \false
          en caso contrario.

    \item Probar que la función o programa es correcto.

    \item Proponer al menos una poda por factibilidad.

    \item Si \ref{ej7:item-b} no tiene una cota superior $O(3^n)$ para la complejidad, analizar el caso donde se separa la
          recursión en tener o no una letra E y ver si mejora la misma.
  \end{enumerate}
\end{enunciado}

Ejercicio de conteo.
\begin{enumerate}[label=$\alph*)$]
  \item Una palabra con guiones bajos que \textit{no tiene soluciones válidas},
        porque no tendría la E, dado que de ponerla tendría 3 vocales consecutivas.:
        $$
          AA\_
          \to
          AA\underline{\blue{c}}
          \flecha{$\paratodo \blue{c} \en \intervalo{a-z}$} \false
        $$
        Una palabra que se puede mejorar sería $A\_\_$, la cual tiene $21 + 21$ soluciones candidatas. Número al que llego
        poniendo una $E$ en el primer guión y cualquier consonante en el segundo o al revés.
        $$
          A\_\_
          \to
          \ub{A\underline{\blue{B}}\_}{\text{solución}\\ \textit{parcial}}
          \to
          A\blue{B}\blue{E} \to \true
        $$

  \item
        Tengo una cadena de caracteres, \textit{un string}, que tiene guiones bajos, hay que meter alguna letra ahí, entonces
        \textit{fuerza bruta}. No sé donde están los guiones bajos, entonces pruebo todo, no sé que letra tengo que poner entonces pruebo
        todas, \textit{fuerza bruta}! Si la letra funciona o no de eso se encargará la función \green{\textit{verificar}}:

        La siguiente función es un algoritmo que reemplaza los \textit{guiones bajos}
        por algún caracter del intervalo $\intervalo{a-z}$.

        Se llama inicialmente con $i = 1$ y el caso base devuelve $0$ o $1$ una vez recorrido todo el \textit{string}.
        $$
          \textit{\green{betterStrings}}(i,n) =
          \llave{ccl}{
            1 & \text{si} & i > n \land \textit{\green{verificar}}(S)  \\
            \textit{\green{betterStrings}}(S[i], i + 1) & \text{si} & i \leq n \land S[i] \distinto \_  \\
            ~\sumatoria{c \en \intervalo{a-z}}{} \textit{\green{betterStrings}}(S[i] \ot c, i + 1) & \text{si} & i \leq n \land S[i] = \_ \\
            0 & \multicolumn{2}{l}{\text{si no}}
          }
        $$
        ¿Te gustó el nombre de la función?

        Potencialmente tengo $n$ guiones bajos. Por cada guión bajo tengo potencialmente $26$ llamadas recursivas:
        El costo de cada subproblema es $O(n)$ porque tengo que revisar todo el \textit{string}. Sea como sea:
        $$
          T(n) = 26T(n-1) + n \entonces T(n) \en O(n \cdot 26^n)
        $$
        Esto podría afinarse un poco más, por ejemplo argumentando que el número debería ser 25, porque estás cambiando una letra, pero dado que
        \ul{no me pidieron que cumpla ningún número de complejidad} lo dejo así.

  \item \Hacer

  \item
        \begin{itemize}
          \item Al trabajar en $S[i]$ verifico que $S[i-1]$ y $S[i+1]$ no sean ambas vocales o consonantes, ya que ahí la llamada recursiva sería
                de 21 caracteres o de 5 en lugar de siempre 26.
        \end{itemize}

  \item \Hacer
\end{enumerate}
