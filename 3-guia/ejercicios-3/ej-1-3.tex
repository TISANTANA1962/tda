\begin{enunciado}{\ejercicio[Equilibrio Digrafo]}
  \textit{Equilibrio Digrafo}

  Demostrar, usando inducción en la cantidad de aristas, que todo digrafo $D$ satisface
  $$
    \sumatoria{v \en V(D)}{} d_{in}(v) =
    \sumatoria{v \en V(D)}{} d_{out}(v) =
    |E(D)|.
  $$
\end{enunciado}

\parrafoDestacado{
  Voy a intercambiar \textit{arcos} y \textit{aristas}, indistintamente, todo el tiempo como disléxico que soy. Se entiende igual.
}

Inducción en las \textit{aristas} del digrafo (grafo dirigido) $D(V,E)$, con $|V| = n \ytext |E| = m$:

Quiero probar que la siguiente proposición $p(m)$ es verdadera:
\begin{center}
  $p(m)$ : Un digrafo $D$, cumple que
  $
    \sumatoria{v \en V}{} d_{in}(v) =
    \sumatoria{v \en V}{} d_{out}(v) =
    m.
  $
\end{center}

\textit{Caso base:}

Quiero probar que cuando $E = \vacio$, lo que implica que $m = 0$,  $p(0)$ es verdadera:
\begin{center}
  $p(\blue{m = 0})$ :  Un digrafo $D$, cumple que
  $
    \sumatoria{v}{} d_{in}(v) =
    \sumatoria{v}{} d_{out}(v) =
    \blue{0}
  $
\end{center}
Si no hay ningún nodo de los $n$, \textit{conectado, unido, adyacente} a ningún otro nodo, los grados $d_{in},\, d_{out}$ son todos 0.
Por lo tanto la proposición $p(0)$ resultó verdadera.

\medskip

\textit{Paso inductivo}:

Supongo que para algún $\blue{k} \en \naturales$ con $\blue{m = k}$ la proposición:
\begin{center}
  $p(\blue{k})$ :
  $
    \ub{
      \text{
        Un digrafo $D$, cumple que
        $
          \sumatoria{v}{} d_{in}(v) =
          \sumatoria{v}{} d_{out}(v) =
          \blue{k}.
        $
      }
    }{
      \text{\purple{hipótesis inductiva}}
    }
  $
\end{center}
\ul{es verdadera}, entonces quiero ver que la proposición
\begin{center}
  $p(\blue{k + 1})$ :
  Un digrafo $D' = (V, E')$, cumple que
  $
    \sumatoria{v \en V(D')}{} d_{in}(v) =
    \sumatoria{v \en V(D')}{} d_{out}(v) =
    \blue{k + 1}.
  $
\end{center}
\ul{también lo sea.}

El digrafo $D'$ tiene \ul{un arco $e$ más en el digrafo $D$} de la \purple{hipótesis inductiva}. Por lo tanto
puedo decir que al haber un arco más habrá un incremento en una unidad en la suma total de los \textit{grados} ($d$),
dado que el arco $e$ debe tener la \textit{cola} y la \textit{cabeza} en algunos nodos $v_t$ y $v_h \en V$:

Escribriendo la sumatoria de la \purple{hipótesis inductiva} abierta:
$$
  \llave{l}{
    \sumatoria{\scriptscriptstyle v \en V(D)}{} d_{in}(v) =
    d_{in}(v_1) + \ldots  + d_{in}(v_n) = \blue{k}    \\
    \sumatoria{\scriptscriptstyle v \en V(D)}{} d_{out}(v) =
    d_{out}(v_1) + \ldots  + d_{out}(v_n) = \blue{k}
  }
$$
Voy a escribir la expresión de la sumatoria de los grados del digrafo $D'$, acomodar para que aparezca la
\purple{hipótesis inductiva}.
$$
  \llave{rcl}{
    \begin{array}{rcl}
      \sumatoria{\scriptscriptstyle v \en V(D')}{} d_{in}(v)
       & =                   &
      d_{in}(v_1) + \cdots + \magenta{d'_{in}(v_h)} + \cdots + d_{in}(v_n)                        \\
       & \igual{\red{!!}}    &
      d_{in}(v_1) + \cdots + \magenta{1 + d_{in}(v_h)} + \cdots + d_{in}(v_n)                     \\
       & =                   &
      \magenta{1} + d_{in}(v_1) + \cdots + \magenta{d_{in}(v_h)} + \cdots + d_{in}(v_n)           \\
       & \igual{\red{!}}     & \magenta{1} +\sumatoria{\scriptscriptstyle v \en V(D)}{} d_{in}(v) \\
       & \igual{\purple{HI}} & \magenta{1} + \blue{k}                                             \\ \\
      \sumatoria{\scriptscriptstyle v \en V(D')}{} d_{out}(v)
       & =                   &
      d_{out}(v_1) + \cdots + \magenta{d'_{out}(v_t)} + \cdots + d_{out}(v_n)
      \igual{\purple{HI}}[idem] \magenta{1} + \blue{k}
    \end{array}
  }
$$

La proposición  $p(\blue{k + 1})$ resultó ser verdadera.

Por lo tanto queda demostrada la proposición $p(m) \paratodo m \en \naturales$ por principio de inducción en los arcos del digrafo.

\fin
