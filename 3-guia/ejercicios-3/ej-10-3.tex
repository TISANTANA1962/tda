\begin{enunciado}{\ejercicio[Triángulo Inductivo]}

  \textit{Triángulo Inductivo}

  Demostrar por inducción que todo grafo de $2n$ vértices con más de $n^2$ aristas tiene algún triángulo.

  ¿Se puede dar una cota mejor que funcione a partir de algún $n_0$? Es decir, ¿Existe $c(n) < n^2$ tal
  que todo grafo de $2n \geq n_0$ vértices con más de $c(n)$ aristas tenga triángulos?
\end{enunciado}

Triángulos:
\parrafoDestacado[\atencion]{
  \textit{
    Dado un vértice $v$, este está conectado a $n$ vecinos formando un conjunto $N(v)$. Para que no haya ningún
    triángulo es necesario que no exista ningúna arista $(u,w), \paratodo u, w \en N(v)$. En otras palabras,
    \ul{
      Los vecinos de $v$ no pueden ser vecinos entre sí.
    }
  }
}

\textit{Un ejemplo ejemplificante con $n = 6,\,|V| = 12$}:

\begin{minipage}{0.5\textwidth}
  \begin{enumerate}[label=\tiny \faIcon{code}$_{\arabic*})$]
    \item Fácil de ver que no hay ningún triángulo siguiendo
          aristas verdes.

    \item Evito que vértices de igual paridad sean vecinos.

    \item Agregar una arista en cualquier lugar conectaría elementos
          de $N(i_{impar})$ con otro de $N(i_{impar})$ y lo mismo con $N(i_{par})$. Generando
          un triángulo.

    \item $d(v) = 6 \paratodo v \en G_6$. Se unen a $\frac{|V|}{2} = \oa{6}{n}$ vértices.

    \item Hay un total de $n^2 = 36 = \frac{12 \cdot 6}{2}$ aristas.

    \item El grafo $G_6$ \textit{bipartito}.
    \item El grafo $G_6$ es un \textit{grafo junta}.
  \end{enumerate}
\end{minipage}
\begin{minipage}{0.5\textwidth}
  $$
    \begin{tikzpicture}[
        grafo style,
        scale=0.4,
        node distance = 1cm,
        minimum size = 0.6cm,
        vecino1/.style={draw=OliveGreen, fill=OliveGreen!20},
        vecino2/.style={draw=Peach, fill=Peach!20},
      ]

      \node[nodo,vecino1] (1) {$1$};
      \node[nodo,vecino2, left = 4cm of 1] (2) {$2$};
      \node[nodo,vecino1, below of = 1] (3) {$3$};
      \node[nodo,vecino1, below of = 3] (5) {$5$};
      \node[nodo,vecino1, below of = 5] (7) {$7$};
      \node[nodo,vecino1, below of =7] (9) {$9$};
      \node[nodo,vecino1, below of = 9] (11) {$11$};
      \node[nodo,vecino2, below of = 2] (4) {$4$};
      \node[nodo,vecino2, below of = 4] (6) {$6$};
      \node[nodo,vecino2, below of = 6] (8) {$8$};
      \node[nodo,vecino2, below of = 8] (10) {$10$};
      \node[nodo,vecino2, below of = 10] (12) {$12$};
      \node[dashed, draw=OliveGreen, fill=OliveGreen!30, fit=(1)(11), inner sep = 10pt, outer sep= 5pt, opacity=0.4, rounded corners] (vecinos1){};
      \node[dashed, draw=Peach, fill=Peach!30, fit=(2)(12), inner sep = 10pt, outer sep= 5pt, opacity=0.4, rounded corners] (vecinos2){};
      \node[] at (vecinos1.north) {$N(i_{par})$};
      \node[] at (vecinos2.north) {$N(i_{impar})$};
      \node[dotted,draw=black, fit=(1)(12), inner sep = 27pt, outer sep= 5pt, rounded corners] (g) {};
      \node[] at (g.north) {$G_6$};

      %1
      \foreach \i in {2,4,6,8,10,12}{
          \draw[OliveGreen, very thick] (1) to (\i);
          \foreach \j in {3,5,7,9,11}{
              \ifnum\i=2
                \draw[Peach, very thick] (2) to (\j);
              \else
                \draw[OliveGreen, very thin] (\j) to (\i);
              \fi
            }
        }
    \end{tikzpicture}
  $$
\end{minipage}

\medskip

¿Medio bardo? Dibujé el grafo 100 veces antes de darme cuenta que era por acá. Bueh, así se puede pensar un grafo de $2n$ vértices con
$n^2$ aristas sin formar ningún triángulo.

Con esta intuición en mente arranco la demo, en la cual voy a tener que probar por absurdo apoyándome en la \purple{hipótesis inductiva}.

\bigskip

Quiero demostrar la siguiente proposición:
\parrafoDestacado{
  \textit{
    $p(n):$ Todo grafo de $2n$ vértices con más de $n^2$ aristas tiene algún triángulo $\paratodo n \en \naturales$.
  }
}

\textit{Caso base:}

Quiero mostrar que la proposición:
\parrafoDestacado{
  \textit{
    $p(\blue{2}):$ Todo grafo de $2\cdot \blue{2}$ vértices con más de $\blue{2}^2$ aristas tiene algún triángulo.
  }
}
es verdadera.

Tengo 4 vértices. Solo puedo forman un único grafo con $\blue{2}^2$ aristas sin triángulo:
$$
  \begin{tikzpicture} [grafo style]
    \node[nodo] (v1) {$v_1$};
    \node[nodo, right of=v1] (v2) {$v_2$};
    \node[nodo, below of=v2] (v3) {$v_3$};
    \node[nodo, left of=v3] (v4) {$v_4$};
    \draw[] (v1) to node[midway, above]{$1$} (v2);
    \draw[] (v2) to node[midway, right]{$2$} (v3);
    \draw[] (v3) to node[midway, below]{$3$} (v4);
    \draw[] (v1) to node[midway, left]{$4$} (v4);
    \draw[dotted, red] (v2) to (v4);
    \draw[dotted, red] (v3) to (v1);
  \end{tikzpicture}
$$
Agregar cualquiera de las aristas \red{punteadas} me formaría los triángulos. Por lo tanto la proposición $p(\blue{2})$ resultó verdadera.

\bigskip

\textit{Paso inductivo:}

Asumo que para algún $\blue{k} \en \naturales$ la proposición:
\parrafoDestacado{
  $p(\blue{k}):
    \ub{
      \textit{
        Todo grafo de $2\blue{k}$ vértices con más de $\blue{k}^2$ aristas tiene algún triángulo,
      }
    }{
      \text{\purple{hipótesis inductiva}}
    }
  $
}
es verdadera, entonces quiero probar que:
\parrafoDestacado{
  \textit{
    $p(\blue{k + 1}):$ Todo grafo de $2(\blue{k + 1})$ vértices con más de $(\blue{k + 1})^2$ aristas tiene algún triángulo,
  }
}
también lo sea.

\textit{Por absurdo}:

Supongo que tengo un grafo $G(V,E)$ con $|V| = 2(\blue{k+1})$ y $|E| \mayor{\red{!!}} (\blue{k+1})^2$ sin ningún triángulo.
\parrafoDestacado{
  \textit{
    Le voy a sacar 2 vértices \red{adyacentes} ($\magenta{v} \ytext \magenta{u}$) a ese grafo $G(V,E)$ para compararlo con la \purple{HI}
  }
}
\parrafoDestacado{
\it
¿Por qué los elijo \red{adyacentes}?

Porque quiero sacar la \ul{menor} cantidad de aristas posibles. Si saco de más siempre voy a poder
conseguir grafo sin triángulo.

{\tiny(Básicamente $p \entonces q$, si $p = \false$, la implicación es \true, pero no aporta nada).}
}
Sacarle 2 vértices a $|V| = 2(\blue{k+1})$ me deja con $|V'| = 2\blue{k}$. Quiero calcular
\parrafoDestacado{
  \textit{
    ¿Cuántas aristas tengo que sacar junto al vértice \magenta{$v$} que elimino?
  }

  \textit{
    ¡La cantidad \red{máxima} que sé que un vértice \magenta{$v$} puede tener sin formar triángulos en el grafo!
  }
}
Laburo haciendo \textit{operaciones} en un \red{grafo genérico} $G$, \textit{operaciones} que no deberían romper la generalidad que debe tener la demo.

En la intro al ejercicio vi que para un vértice $\magenta{v}$ que tiene un conjunto de vecinos $N(\magenta{v})$
es necesario que los elementos $w \en N(\magenta{v})$ no sean vecinos entre sí.
\parrafoDestacado{
  \textit{
    Laburo con esos vértices $\magenta{v}$, los cuales \ul{debe} haber en cualquier grafo con $|E| > (\blue{k+1})^2$.
    Recuerdo que asumí que \ul{no} hay triángulos y esos vértices que quiero sacar $\magenta{u}$ maximizan la cantidad de
    aristas que puede tener un vértice \ul{sin formar triángulos}.
  }
}

Saco el vértice \magenta{$v$} \ul{primero}, \ul{luego} el \magenta{$u$}. Ayuda mucho el grafo que grafiqué en la intro
para ver la aristas afectadas al sacar los vértices 1 y 2:
\begin{itemize}
  \item El vértice \magenta{$v$} está unido a otros $\frac{2(\blue{k+1})}{2} = \ob{\blue{k + 1}}{|N(\magenta{v})|}$ vértices (incluyendo al \magenta{$u$}).
  \item ¡Listo, ya saqué el \magenta{$v$} junto a sus $\blue{k+1}$ aristas! (tantas aristas como la mitad de vértices).
  \item El vértice $\magenta{u}$ (ahora) está unido a otros $\ob{\frac{2(\blue{k + 1})}{2} - \ua{1}{\text{el} ~ \magenta{v} ~ \text{que}\\ \text{se fue}}}{|N(\magenta{u}) \setminus \set{\magenta{v}}|} = \blue{k}$.
  \item En total le saco $\blue{2k+1}$ aristas a $G$.
\end{itemize}

Recuerdo que: $G(V,E)$ tiene $|E| > (\blue{k+1})^2$ y \red{no tiene triángulos}.

Mi nuevo grafo podado $G'$:
$$
  \llamada1
  G'(V', E') ~ \text{con} ~
  \llave{rcl}{
    |V'| & = & 2\blue{k} \\
    |E'| & = & |E| - (\blue{2k + 1}) \mayor{\red{!!!}} \blue{k}^2
  }
$$
\parrafoDestacado[\atencion]{
  \textit{
    ¿Ves al absurdo?
  }
}
Partiendo de un grafo $G$ con $2(\blue{k+1})$ vértices y más de $(k+1)^2$ aristas supuestamente \red{sin} triángulos, llegué
\ul{con el cuidado necesario de realizar operaciones que no destruirían ningún triángulo}
a un grafo $\llamada1G'$ con $2k$ vértices y más (\red{!!!}) de $k^2$ aristas \red{sin triángulos tampoco}.
Un minuto de silencio, porque acaba de morir la \purple{hipótesis inductiva}: Ahí está el absurdo.

\bigskip

Por el absurdo sale entonces que la proposición $p(\blue{k+1})$ también es verdadera, dado que no puedo tener un grafo $G$ con $2(k+1)$ vértices y
$(k+1)^2$ aristas sin triángulos.

Por el principio de inducción $p(n)$ será verdadera $\paratodo p \en \naturales$.

\bigskip

Sobre la pregunta de si hay una cota, creo que no, por la forma en que construyo el grafo de la intro, es eso una demo? Sé que no.

No sé, ni quiero demostrarlo. Si podés demostrarlo y coso escribime y lo corrijo o agrego.

\fin
