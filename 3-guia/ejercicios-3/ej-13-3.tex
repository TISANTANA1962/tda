\begin{enunciado}{\ejercicio[Grafo conexo tiene dos vértices que no son de articulación]}

  \textit{Grafo conexo tiene dos vértices que no son de articulación}

  Todo $G_n (n \geq 2)$ conexo tiene al menos dos vértices distintos $v_1, v_2$ tal que
  $G \diferencia \set{v_1}$ y $G \diferencia \set{v_2}$ son conexos.
\end{enunciado}

Quiero probar que la proposición
\parrafoDestacado{
  \it
  $p(n)$: Todo $G(V,E)$ con $|V| = n$ conexo tiene al menos dos vértices distintos $v_1, v_2$ tal que
  $G \diferencia \set{v_1}$ y $G \diferencia \set{v_2}$ son conexos.
}
es verdadera para todo $n \en \naturales_{\geq 2}$

\bigskip

\textit{Caso base:}

Quiero probar que la proposición
\parrafoDestacado{
  \it
  $p(\blue{2})$: Todo $G(V,E)$ con $|V| = \blue{2}$ conexo tiene al menos dos vértices distintos $v_1, v_2$ tal que
  $G \diferencia \set{v_1}$ y $G \diferencia \set{v_2}$ son conexos.
}
es verdadera.

Dado que el un grafo de un solo vértice es trivialmente conexo la proposición $p(\blue{2})$ es verdadera.

\bigskip

\textit{Paso inductivo:}

Asumo que para algún $\blue{k} \en \naturales_{\geq 2}$ la proposición
\parrafoDestacado{
  \it
  $p(\blue{k})$:
  $
    \ob{
      \textit{ Todo $G(V,E)$ con $|V| = \blue{k}$ conexo tiene al menos dos vértices distintos}
    }{
      $\purple{hipótesis inductiva}$
    }
  $
  $v_1, v_2$ tal que
  $G \diferencia \set{v_1}$ y $G \diferencia \set{v_2}$ son conexos.
}
es verdadera. Quiero probar entonces que
\parrafoDestacado{
  \it
  $p(\blue{k+1})$: Todo $G(V,E)$ con $|V| = \blue{k+1}$ conexo tiene al menos dos vértices distintos $v_1, v_2$ tal que
  $G \diferencia \set{v_1}$ y $G \diferencia \set{v_2}$ son conexos.
}
también lo sea.

Partiendo de $G(V, E)$ con $|V| = \blue{k + 1}$ tengo estos casos
\begin{itemize}
  \item[\grafoBullet] Este es el caso fácil con viento a favor:

        Saco un $v$ \ul{cualquiera y obtengo un subgrafo conexo}, es decir un $G_1 = G(V \setminus \set{v}, E')$,
        y para otro vértice $u$ obtendría otro subgrafo
        $G_2 = G(V \setminus \set{u}, E'')$. Probando así que un grafo conexo tiene por lo menos dos vértices que no son
        de articulación.

        % De ocurrir esto, entonces seguro puedo encontrar otro vértice $v_2$ en el grafo unión $G' \union \set{v_1}$ en esa componente
        % de manera de obtener un grafo conexo, ya que estoy en un grafo con $|V| = \blue{k}$ y por \purple{hipótesis inductiva} dicho vértice existe.
        %
        % Si bien saqué los vértices de forma secuencial, queda claro que partiendo del $G$ original voy a poder encontrar al menos dos vértices
        % distintos $\set{v_1, v_2} \en V_G$ tal que los grafos obtenidos sean conexos.
        % $$
        %   \begin{tikzpicture}[grafo style]
        %     \node[nodo, label=above:{$v_1$}] (v) {};
        %     \node[nodo, right of = v] (ci) {$c_i$};
        %     \node[nodo, above of = ci] (c1) {$c_1$};
        %     \node[nodo, below of = ci] (c2) {$c_2$};
        %
        %     \draw[] (v) to (c2) to (ci) to (c1) to (v) to (ci);
        %     \node[compConexa, fit=(v)(c1)(c2), label=above:{$G(V,E)$}](g1) {};
        %
        %     \begin{scope}[xshift=4cm, yshift=2cm]
        %       \node[nodo,opacity=0.2] (v) {};
        %       \node[nodo, right of = v] (ci) {$c_i$};
        %       \node[nodo, above of = ci] (c1) {$c_1$};
        %       \node[nodo, below of = ci] (c2) {$c_2$};
        %
        %       \draw[] (c2) to (ci) to (c1) to (ci);
        %       \node[compConexa, fit=(c1)(c2)(v), label=above:{$G(V\setminus \set{v_1},E')$}, inner sep = 10pt] (g2){};
        %       \node[compConexa, thin, draw=blue, fit=(c1)(c2), inner sep = 5pt] (conexa) {};
        %       \node[rectangle, draw = BrickRed, above right = 1cm of g2, anchor=west,text width=2.5cm, font=\tiny] (v2)
        %       {Acá debe haber un \inlinegraph{v_2} que no rompa la conexidad por \purple{HI}};
        %       \draw[-latex, bend left = 30pt] (v2.south) to (conexa.east) {};
        %     \end{scope}
        %
        %     \begin{scope}[xshift=7cm, yshift=-2cm]
        %       \node[nodo,label=above:{$v_1$}] (v) {};
        %       \node[rectangle, rounded corners=8pt, draw=BrickRed,fill=BrickRed!5, right of = v] (ci) {$c_i\setminus \set{v_2}$};
        %       \node[nodo,above of = ci] (c1) {$c_1$};
        %       \node[nodo, below of = ci] (c2) {$c_2$};
        %       \draw[] (v) to (c1)  (v) to (c2) (v) to (ci) (v) to (c1) (ci) to (c1) (ci) to (c2);
        %       \node[compConexa, fit=(c1)(c2)(v)(ci), label=above:{$G(V\setminus \set{v_2},E'')$}, inner sep = 10pt] (g3){};
        %     \end{scope}
        %
        %     \draw[-latex, bend right = 30pt] (g2.south) to (g3) {};
        %     \draw[-latex, bend right = 20pt] (g1) to (g3) {};
        %     \draw[-latex] (g1) to (g2);
        %   \end{tikzpicture}
        % $$
        %
        % Que bardo eso que hice, creo que confunde más que aclarar.
        % Pero intuitivamente va por ahí, deben existir esos \textit{wuachines $v_1$} y \textit{$v_2$}, de forma tal que los subgrafos
        % sean sean conexos. {\tiny Cuando tenga ganas veo de \textit{remasterizarlo}}.

  \item[\grafoBullet]
        Este es el caso más genérico sin viento a favor:

        Saco un $v$ cualquiera de $G(V,E)$ y aparecen \ul{por lo menos} 2 componenetes conexas $\set{c_1,\ldots,c_i}$,
        es decir que $v$ es un vértice de articulación, \textit{no lo puedo sacar para probar la proposición \red{¡Tengo que encontrar otros!}}

        Armo un grafo para ver si puedo usar la \purple{hipótesis inductiva}:

        Agarrando alguna de las componenetes conexas, $c_j$ que aparecieron y \ul{reconectándola} al $v$, formo
        un grafo $G_j(V',E')$ con $|V'| \geq 2$ que ¡Cumple la \purple{hipótesis inductiva}!

        $G_j(V',E')$ tiene \ul{por lo menos 2 vértices} $\set{v, u,\ldots}$, ese $u$ (\red{que no es $v$}) lo agarro para sacárselo al $G$ original:
        $$
          \begin{tikzpicture}[grafo style]
            \node[nodo, label=above:{$v$}] (v) {};
            \node[nodo, right of = v] (ci) {$c_i$};
            \node[nodo, above of = ci] (c1) {$c_1$};
            \node[nodo, below of = ci] (c2) {$c_x$};

            \draw[] (v) to (c2) (v) to (ci) (v) to (c1);
            \node[compConexa, fit=(v)(c1)(c2), label=above:{$G(V,E)$}](g1) {};

            \begin{scope}[xshift = 4cm]
              \node[nodo, label=above:{$v$}] (v) {};
              \node[nodo, right of = v] (ci) {$c_j$};
              \node[nodo, above of = ci, opacity=0.4] (c1) {$c_1$};
              \node[nodo, below of = ci, opacity=0.4] (c2) {$c_x$};

              \draw[] (v) to (ci);
              \node[compConexa, thin, blue, fit=(v)(ci), label=left:{$G_j(V', E')$}, inner sep = 5pt] (cv) {};
              \node[rectangle, draw = BrickRed, above right = 1cm of c1, anchor=west,text width=2.5cm, font=\tiny] (v2)
              {Acá hay un \inlinegraph{u} que si lo saco no rompe la conexidad por \purple{HI}};
              \draw[-latex, bend left = 30pt] (v2.south) to (cv.east) {};
            \end{scope}

            \begin{scope}[xshift = 8cm, yshift = -2cm]
              \node[nodo, label=above:{$v$}] (v) {};
              \node[nodo, rectangle, rounded corners=6pt, draw=BrickRed,fill=BrickRed!5,right of = v] (ci) {$c_j \setminus \set{u}$};
              \node[nodo, above of = ci] (c1) {$c_1$};
              \node[nodo, below of = ci] (c2) {$c_x$};

              \draw[] (v) to (c2) (v) to (ci) (v) to (c1);
              \node[compConexa, fit=(v)(c1)(c2)(ci), label=above:{$G(V\setminus\set{u},E')$}](g3) {};
              % \node[rectangle, draw = BrickRed, above right = 1cm of c1, anchor=west,text width=2.5cm, font=\tiny] (v2)
              % {Acá debe haber un \inlinegraph{u} distinto de $v$ que no rompa la conexidad por \purple{HI}};
              % \draw[-latex, bend left = 30pt] (v2.south) to (cv.east) {};
              \draw[-latex, bend right = 20pt] (cv.south east) to (g3) {};
              \draw[-latex, bend right = 20pt] (g1) to (g3) {};
              % \draw[-latex] (g1) to (g2);
            \end{scope}
          \end{tikzpicture}
        $$

        Aplicar el mismo razonamiento en otra de las componentes conexas serviría para encontrar un $w$ (\red{que no sería ni $v$ ni $u$}),
        así probando que hay por lo menos 2 vértices que puedo sacarle a un grafo conexo de $\blue{k+1}$ vértices, de forma tal de armar 2 subgrafos conexos.
\end{itemize}

Dado que $p(2), p(k) \ytext p(\blue{k+1})$ resultaron verdaderas, también lo es $p(n) \en \naturales_{\geq 2}$ por principicio de inducción.
\fin
