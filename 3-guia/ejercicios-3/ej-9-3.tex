\begin{enunciado}{\ejercicio[Unicidad de Grados]}
  \textit{Unicidad de Grados}

  Sean $G_2 = K_2 \ytext G_{n+1} = \overline{G_n \union K_1}$ para todo $n \geq 2$. Demostrar por inducción que $G_n$ tiene un único
  par de vértices de igual grado.
\end{enunciado}

Dado el grafo definido por recurrencia:
$$
  \llave{rcl}{
    G_2 & = & K_2\\
    G_{n+1} & = & \overline{G_n \union K_1}
  }
$$

Quiero demostrar la siguiente proposición:

\parrafoDestacado{
  \textit{
    $p(n) :$ $G_n$ tiene un único par de vértices de igual grado $\paratodo n \en \naturales$.
  }
}

\bigskip

\textit{Caso base:}

\parrafoDestacado{
  \textit{
    $p(\blue{3}) :$ $G_{\blue{3}}$ tiene un único par de vértices de igual grado.
  }
}

\parrafoDestacado[\atencion]{
  En el ejercicio \refEjercicio{ej:8} se definió al \textit{grafo unión}, el cual particularmente, tiene la misma cantidad de aristas que
  la suma de las aristas de los grafos (disjuntos) que se están uniendo.

  Cuando se calcula el complemento de un grafo $G(V,E)$,
  la cantidad de aristas de $|E(\bar{G})| = \frac{|V| \cdot (|V|-1)}{2} - |E|$.
}

Como sé que por definición:
$$
  G_{\blue{3}} \igual{def} ~ \overline{G_2 \union K_1} ~ \flecha{calculo}[aristas] ~ |E(G_{\blue{3}})| = \frac{\blue{3} \cdot (\blue{3} - 1)}{2} - 1 = 2
$$

$G_{\blue{3}}$ tiene 3 vértices y 2 aristas. Es un grafo conexo obligatoriamente.
\parrafoDestacado{
  \textit{¿Cómo puedo armar ese grafo?} Así:
}
$$
  \llave{rcl}{
    d(v_1) & = & 1 \\
    d(v_2) & = & 1 \\
    d(v_{\blue{3}}) & = & 2
  }
  \qquad
  \begin{tikzpicture}[grafo style, baseline = 20]
    \begin{scope}[xshift = 5cm]
      \node[nodo] (v3) {$v_{\blue{3}}$};
      \node[nodo, above left of = v3] (v1) {$v_1$};
      \node[nodo, above right of = v3] (v2) {$v_2$};
      \draw[arista, thick, OliveGreen] (v3) to (v1);
      \draw[arista, thick, OliveGreen] (v3) to (v2);
    \end{scope}

    \draw[-Latex, thick, black] (2, 0.8) -- ++(1, 0) node[midway, below]{complemento};

    \node[nodo] (v3) {$v_{\blue{3}}$};
    \node[nodo, above left of = v3] (v1) {$v_1$};
    \node[nodo, above right of = v3] (v2) {$v_2$};
    \draw[arista, thick, black] (v1) to node[midway, above] {$K_2$} (v2);
  \end{tikzpicture}
$$

{
    \footnotesize
    \color{gray!25}(¡Bueh, que larga que la hice! Pero quería pasear por un poco, entrando en calor para el paso inductivo.)
  }

Ahí queda demostrado que la proposición $p(\blue{3})$ es verdadera.

\bigskip

\bigskip

\textit{Paso inductivo:}

\ul{Asumo} que para algún $\blue{k} \en \naturales$ la proposición
\parrafoDestacado{
  $p(\blue{k}) :$
  $
    \ub{
      G_{\blue{k}} ~ \textit{tiene un único par de vértices de igual grado.}
    }{
      \text{\purple{hipótesis inductiva}}
    }
  $
}
es \ul{verdadera}. Entonces quiero mostrar que
\parrafoDestacado{
  \textit{
    $p(\blue{k + 1}) :$ $G_{\blue{k + 1}}$ tiene un único par de vértices de igual grado.
  }
}
también lo sea.

\medskip

Estos podrían ser unos lemitas que no pienso demostrar:
\parrafoDestacado[$\llamada1$]{
  \textit{
    Si en un grafo $G$, el grado de un vértice $v \en V(G)$ es $d_{G}(v)$, ese mismo vértice en $\bar{G}$ tendrá un grado
    cumpliendo que
    $d_{\bar{G}}(v) + d_{G}(v) = |V| - 1$.
    Es prácticamente la definición de grafo complemento.
  }
}
Se desprende fácil de esa última afirmación:
\parrafoDestacado[$\llamada2$]{
  \textit  {
    Si $G$ no es un grafo conexo, entonces $\bar{G}$ sí lo es.
  }
}

\bigskip

Recopilando cosas que se vieron hasta acá y con la info de la \purple{hipótesis inductiva}, sé que:
\begin{itemize}
  \item Por la definición recursiva de $G_{\blue{k+1}}$ es un \textit{grafo junta}, es decir que su conjugado
        $\bar{G}_{\blue{k+1}}$ es un \textit{grafo unión}.

  \item $G_{\blue{k}}$ tiene \ul{únicamente} 2 vértices de igual grado (\purple{HI}).

  \item Cuando hago la unión $G_{\blue{k}} \union K_1$ tengo un grafo disconexo con $\blue{k + 1}$ vértices, luego calculo
        $$
          G_{\blue{k + 1}} = \overline{G_{\blue{k}} \union K_1}
        $$
        el complemento de esa unión y obtengo un grafo conexo ($\llamada2$).

  \item $G_{\blue{k}}$ por definición será un grafo conexo
        (i.e. $d_{G_{\blue{k}}}(v) \distinto 0 \paratodo v \en G_{\blue{k}}$).
        \ul{$d_{G_{\blue{k}}}(v) \en [1,\blue{k}-1] \paratodo v \en G_{\blue{k}}$}.

  \item Los grados del grafo $G_{\blue{k+1}}$ se pueden calcular a partir de los grados del grafo $G_{\blue{k}}$ usando $\llamada1$:
        $$
          d_{G_{\blue{k+1}}}(v) =
          \llave{cll}{
            \blue{k} &\text{si}& v = v_{\blue{k+1}} ~ (\text{el que agrega} ~ K_1) \\
            \ub{\blue{k+1}}{\scriptscriptstyle|V(G_{\blue{k+1}})|} - 1 - d_{G_{\blue{k}}}(v) & \multicolumn{2}{l}{\text{si no}}
          }
        $$

        \parrafoDestacado{
          \textit{¿Qué está pasando?}
        }

  \item Los valores de $d_{G_{\blue{k+1}}}(v)$ solo van a repetirse para valores iguales de $d_{G_{\blue{k}}}(v)$
        dado que es una función biyectiva en $d_{G_{\blue{k}}}(v)$, más aún por \purple{hipótesis inductiva} $d_{G_{\blue{k}}}(v)$
        tiene solo \ul{dos valores repetidos}, implicando que solo habrá \ul{dos valores iguales para el conjunto $d_{G_{\blue{k+1}}}(v)$}.
\end{itemize}

Así tengo que $p(\blue{k+1})$ resultó ser verdadera también. Por el principio de inducción $p(n)$ es verdadera $\paratodo n \en \naturales_{\geq 2}$.
\fin
